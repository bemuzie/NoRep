

%%% Поля и разметка страницы %%%
\usepackage{lscape}		% Для включения альбомных страниц
\usepackage{geometry}	% Для последующего задания полей

%%% Удобные инклуды
\usepackage{catchfilebetweentags} % синтаксис: %<*tag> ....   ....  %</tag>



%%% Кодировки и шрифты %%%
\usepackage{cmap}						% Улучшенный поиск русских слов в полученном pdf-файле
\usepackage[T2A]{fontenc}				% Поддержка русских букв
\usepackage[utf8]{inputenc}				% Кодировка utf8
\usepackage[russian]{babel}	% Языки: русский, английский
%\usepackage{cyrtimes}						% Красивые русские шрифты


%%% Математические пакеты %%%
\usepackage{amsthm,amsmath,amssymb,amscd} % Математические дополнения от AMS

%%% Оформление абзацев %%%
\usepackage{indentfirst} % Красная строка

%%% Цвета %%%
\usepackage[usenames]{color}
\usepackage{color}
\usepackage{colortbl}

%%% Таблицы %%%
\usepackage{longtable}					% Длинные таблицы
\usepackage{multirow,makecell,array}	% Улучшенное форматирование таблиц
\usepackage{floatrow}           
\floatsetup[table]{capposition=top,style=Plaintop} % floatrow перемещение заголовка таблицы вверх
\usepackage{rotating} % поворот заголовков таблиц (используется xtable)


%%% Общее форматирование
\usepackage[singlelinecheck=off,center]{caption}	% Многострочные подписи
\usepackage{soul}									% Поддержка переносоустойчивых подчёркиваний и зачёркиваний

%%% Библиография %%%
\usepackage{cite} % Красивые ссылки на литературу

%%% Гиперссылки %%%
\usepackage[linktocpage=true,plainpages=false,pdfpagelabels=false, unicode]{hyperref}

%%% Изображения %%%
\usepackage[pdftex]{graphicx} % Подключаем пакет работы с графикой
%\usepackage{subfigure}
\usepackage{subfig}
%\usepackage{tikz} % библиографические ссылки в графиках

%%% Оглавление %%%
%\usepackage[subfigure]{tocloft}

%%%Аннотации
\usepackage{mparhack}

%%%Для переноса номера страницы вверх
\usepackage{fancyhdr} % Taken from http://tex.stackexchange.com/questions/88574/page-numbering-at-the-top-centre

%%% Список сокращений
\usepackage[acronym,nomain,nonumberlist]{glossaries}
\makeglossaries


%Размер шрифта заголовков
\usepackage{titlesec}
