

%%% Макет страницы %%%
\geometry{a4paper, mag=1000,
          left=2.5cm, right=1cm, top=2cm, bottom=2cm, headsep=0.7cm, footskip=1cm}

%%% Кодировки и шрифты %%%
%\renewcommand{\rmdefault}{ftm} % Включаем Times New Roman
\renewcommand{\rmdefault}{cmr} % Включаем Times New Roman

%%% Выравнивание и переносы %%%
\sloppy					% Избавляемся от переполнений
\clubpenalty=10000		% Запрещаем разрыв страницы после первой строки абзаца
\widowpenalty=10000		% Запрещаем разрыв страницы после последней строки абзаца

%%% Интервалы
\linespread{2}                    % Полуторный интервал (ГОСТ Р 7.0.11-2011, 5.3.6)

%%% Абзац
\setlength{\parindent}{1.27cm}  % абзацный отступ 5 знаков (ГОСТ Р 7.0.11-2011, 5.3.7)


%%% Библиография %%%
\makeatletter
\bibliographystyle{./ugost2003}	% Оформляем библиографию в соответствии с ГОСТ 7.0.1-2003
\renewcommand{\@biblabel}[1]{#1.}	% Заменяем библиографию с квадратных скобок на точку:
\makeatother

%%% Изображения %%%
\graphicspath{ {images/}{articles/PerCTPanCr/images/}{articles/image_quality/images/}{../cases/}{../../../cases/} } % Пути к изображениям
% \usepackage[font=small,labelfont=bf,tableposition=top]{caption} % Уменьшение размера подписи к рисункам
\renewcommand{\thesubfigure}{\asbuk{subfigure}} % subref по-русски
\captionsetup[figure]{justification=justified,singlelinecheck=false} %выключка подписей к рисункам влево
\captionsetup[subfigure]{justification=centering}

%%% Цвета гиперссылок %%%
%\definecolor{linkcolor}{rgb}{0.9,0,0}
%\definecolor{citecolor}{rgb}{0,0.6,0}
%\definecolor{urlcolor}{rgb}{0,0,1}
\hypersetup{
    colorlinks, linkcolor={linkcolor},
    citecolor={citecolor}, urlcolor={urlcolor}
}

%%% Оглавление %%%
\renewcommand{\cftchapdotsep}{\cftdotsep}

%%% Нумерация
%\renewcommand{\thefigure}{\arabic{figure}}
\usepackage{chngcntr}
\counterwithout{figure}{chapter}
\counterwithout{table}{chapter}

%%%Сноски на странице
\newcommand{\annotation}[1]{%
  %\marginpar{\small\itshape\color{blue}#1} % делаем их невидимыми
  }

%%%Для переноса номера страницы вверх
% Taken from http://tex.stackexchange.com/questions/88574/page-numbering-at-the-top-centre
      \fancyhf{}
      \fancyhead[C]{\thepage}
      \pagestyle{fancy}
      
      % redefine the plain pagestyle
      %\let\ps@plain\ps@fancy% Plain page style = Fancy page style
            \fancypagestyle{plain}{%
      \fancyhf{} % clear all header and footer fields
      \fancyhead[C]{\thepage} % except the center
      }
      
      
      \renewcommand{\headrulewidth}{0pt} %% remove fancyhdr line
      

% Шрифт заголовков %(ГОСТ Р 7.0.11-2011, 5.3.5)
\titleformat{\section}
  {\bfseries\centering\fontsize{14}{14}}{\thesection}{1em}{}
\titleformat{\subsection}
  {\centering\fontsize{14}{14}}{\thesection}{1em}{}  


\titlespacing{\section}{0pt}{3ex}{3ex}   % Заголовки должны быть отделены от текста сверху и снизу тремя интервалами
\titlespacing{\subsection}{0pt}{3ex}{3ex}
\titlespacing{\chapter}{0pt}{3ex}{3ex}

%%% Списки %%%
% Используем дефис для ненумерованных списков (ГОСТ 2.105-95, 4.1.7)
\renewcommand{\labelitemi}{\normalfont\bfseries{--}} 

\DeclareCaptionLabelFormat{gostfigure}{Рисунок #2}
\DeclareCaptionLabelFormat{gosttable}{Таблица #2}
\DeclareCaptionLabelSeparator{gost}{~---~}
\captionsetup{labelsep=gost}
\captionsetup[figure]{labelformat=gostfigure}
\captionsetup[table]{labelformat=gosttable}